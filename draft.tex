\documentclass[12pt]{article}
\usepackage{polyglossia}
\setdefaultlanguage{english}
\usepackage{acl}
\usepackage{amsmath}
\usepackage{geometry}
\usepackage{hyperref}
\usepackage{multirow}
\usepackage{fontspec}
\usepackage[T2A]{fontenc}
% \setmainfont[Ligatures=TeX]{Times New Roman}
\setmainfont[Ligatures=TeX]{Linux Libertine O}
\usepackage{microtype}
\usepackage{latexsym}

\title{A Formal Analysis of Classical Greek Conjugation}
\author{Ethan Yates\\
        Brandeis University\\
        Waltham, MA\\
        ethanyates@brandeis.edu}
\date{2022}

\begin{document}
\newcommand{\koronis}[1]{#1\symbol{"0313}}
\newcommand{\dasia}[1]{#1\symbol{"0314}}
\newcommand{\macronbelow}[1]{#1\symbol{"0331}}
\newcommand{\cedilla}[1]{#1\symbol{"0326}}
% \newcommand{\rpsi}{p\textsubscript{s}}
\maketitle

\begin{abstract}
    This article is an attempt to formalize the Greek conjugational system with
    mathematical notation as a means of working towards a computational
    implementation of the synthesis of verb forms. It is hoped that such a tool
    will be of use to students who are learning the nuances of the complex
    verbal system in classical Greek. With this attempt at formalization, it is
    hoped that it will provide the core basis for the development of a
    morphologically intelligent information retrieval system for classical
    Greek utiziling modern analyses and tools.
\end{abstract}

\section{Introduction}

Classical Greek has been studied extensively for thousands of years. Reading
knowledge of classical Greek (henceforth referred to just as \textit{Greek}) is
a sought-after skill in the humanities, especially in the fields of classical
studies, theology, philosophy, and linguistics. As a result of its long
history, the grammar of classical Greek has been very thoroughly studied.
However, the main practitioners of classical Greek work outside of the field of
linguistics (classical studies, theology, and philosophy) meaning the amount the
most modern methods in linguistics have yet to be used to analyze the language to
the same degree as modern languages.

One of the most important resources for studying Greek is the
\textit{Perseus Digital Library}, a platform that hosts ancient texts (in both
Latin and Greek) in a reading environment. The platform allows users to click
on individual words in order to show their lemma and morphological analysis. 

Morphologically intelligent information retrieval (expecially verbs) is of
utmost importance to students of classical Greek (hereafter referred to just as
\textit{Greek}). Verb conjugations are notoriously difficult for students in
their first year of study, due to the relatively high complexity of the
conjugational system in comparison to modern languages more widely spoken
Europe and North America with simpler verbal paradigms. The conjugational
system utilizes thematic vowels, concatenation of personal endings,
reduplication, and derivational prefixes, resulting in around 2 million
potential forms for each verb \citep{crane91}. Most textbooks and reference
grammars have examples of

There is in fact already a program that does this--\textit{Morpheus} is a
program created by Perseus Digital Library to take Greek words as input and
output their morphological analysis. Morpheus was used to create the treebanks
of Greek texts hosted by Perseus. These treebanks (organized by individual
text) contain texts organized by sentence, with each word tagged for its
morphology and dependency syntax. While they have quite extensive coverage of
the most famous Greek texts, not every text has been analyzed. Classical Greek,
having a written history spanning thousands of years, has many texts which have
not been digitzed from old printed editions, and even many texts that have
never been edited into print from the original manuscripts. 

Morpheus was written in the C programming language decades ago, using a very
rigorous set of rules written by classical scholars. A new program written in a
language more suitable for data manipulation (such as Python) is desireable.
The amount of annotated data in the treebanks is enough to work towards
building a model using statistical methods.  However, given the relatively
large amount of features for each verb form, it is necessary to first formulate
a computational analysis of Greek conjugation to allow for the generation of
verb forms, given a lemma and a set of features as input. Once it is possible
to generate verb forms, it is an important step towards building a system that
works in the \textit{opposite} direction, to parse the morphological features
and the lemma from a given verb form. This paper will focus on building a
formal analysis of Greek conjugation (inspired by previous work on other
languages) as a first step toward building a program that will synthesize
verb forms and provide a basis for a larger system. In addition, it will be
demonstrated how such synthesis of verb forms can be used to build corpora 
for statistical analysis.

\section{Prior work on Conjugation}

\subsection{Linguistic treatments of conjugation}

Greek grammar has a very long tradition stretching back to antiquity. The
\textit{modern} study of Greek grammar dates back c. ~200 years. In this
computation analysis, it is desireable to use the most thorough analysis of the
conjugational system as possible, while drawing potential insights from older
sources. In the English language, the pioneering reference grammar is \textit{A
Greek Grammar for Colleges} \citep{smyth1920} which describes the entire
language in detail.  Almost a century later, Cambridge University published
\textit{The Cambridge Grammar of Classical Greek} \citep{boas2019}, which
represents the most "up to date" reference Grammar published in English.  The
system presented in this paper is based on the sections on verb inflection in
both reference grammars. 

Both reference grammars provide a mostly \textit{synchronic} analysis (dipping
into diachronic analysis in some areas to give context). Given that historical
linguistics often provides explanation for so-called "irregular" forms (in
traditional classroom pedagogy), it was necessary to consult \textit{The
Origins of the Greek Verb} \citep{willi2018}, which provides the most thorough
analysis of Greek conjugation in the context of Indo-European historical
linguistics.

In 1948, Roman Jakobson published a famous alaysis of Russian conjugation in
which he breaks the verbal system down into a system of rules (in prose form)
\citep{jakobson48}. This approach was well recieved and emulated by other
authors in analyses of Polish \citep{schenker1954}, Czech
\citep{rubenstein1951}, and Old Church Slavonic/Old Russian \cite{halle1951}.

\subsection{Computational treatments of conjugation}

Multiple scholars were inspired by Jakobson's rule-based approach. A
computational implementation of \citet{jakobson48} was written in the ALGOL60
programming language \citep{kortlant1971}. Mathematician Joachim Lambek,
noticing how a rule-based system lends itself to a mathematical analysis, wrote
a computational analysis of French conjugation \citep{lambek75}. The main 
development by Lambek was the formalization of a rule-based system like in
\citet{jakobson48}, into rewrite rules, lending themselves to sting manipulation.

As mentioned before, \textit{Morpheus} was developed for morphologically
intelligent information retrieval utilizing an extensive rule--based system
curated by experts \citep{crane91}. This program both parses and generates
forms of words, and has been used to generate large annotated corpora. These
rules are extremely numerous, and designed for multiple dialects.  The program
is quite old (written in \textit{C}) and does not utilize statistical methods.
Working towards a new program for morphologically intelligent information
retrieval is the main motivation for this paper.

\section{Scope of analysis}

The full paradigm of a Greek verb contains up to 1000 distinct forms, and when
including derivational prefixes, can number in the millions \citep{crane91}.
Therefore, it is necessary to lessen the scope of the analyses of verbs. In
this article, only the \textit{indicative} voice of the \textit{regular} verb
(terms to be explained later) will be analyzed in order to create the
beginnings of a system that can be expanded to cover all types of Greek verbs.
This paper will first give as concise a \textit{linguistic} description of
conjugation as is possible, followed by a \textit{mathematical} analysis of the
conjugational system. The end of the article will contain some proposals as to
how the data from this program could be utilized.

\section{Linguistic overview of Greek conjugation}

\subsection{Features of Verbs}

When it comes to morphologically intelligent information retrieval, it is
necessary to define what information is being retrieved (or in the opposite
case of the synthesis of verb forms, what information is given to the function
as input). There are \textbf{five} features to be extracted. Verbs are marked
for \textit{person}, \textit{number}, \textit{tense}, \textit{voice}, and
\textit{mood}. Each potential option for these features is given below.

\begin{itemize}
    \item \textbf{Person}: \textit{first, second, third}
    \item \textbf{Number}: \textit{singular, dual, plural}
    \item \textbf{Tense}: \textit{present, imperfect, future, aorist, perfect, pluperfect.}
    \item \textbf{Voice}: \textit{active, middle, passive}
    \item \textbf{Mood}: \textit{indicative, subjunctive, optative}
\end{itemize}

Not all tense-mood combinations exist--Table \ref{tab:tense-mood} shows which
combinations have conjugations. Traditional pedagogy designates a
\textit{sequence of tenses}, dividing tenses into \textit{primary} and
\textit{secondary} tenses (in modern linguistic literature called
\textit{non-past} and \textit{past}, such as in \citet{willi2018}). This
classification determines which prefixes and personal endings are used when
selecting morphemes. These ``tenses" are in fact \textbf{tense-aspect}
combinations, and a tense-aspect combination will be referred to as a
\textbf{tense-system}.

\begin{table*}
\centering
\begin{tabular}{|c|c|c|c|c|c|}
    \hline
        & \textbf{Indicative} & \textbf{Subjunctive} & \textbf{Optative} & \textbf{Imperative} & \textbf{Sequence}\\    
    \hline
    \textbf{Present}    & + & + & + & + & \textbf{primary}\\ 
    \hline
    \textbf{Imperfect}  & + & - & - & - & \textbf{secondary}\\ 
    \hline
    \textbf{Future}     & + & - & + & - & \textbf{primary}\\ 
    \hline
    \textbf{Aorist}     & + & + & + & + & \textbf{secondary}\\ 
    \hline
    \textbf{Perfect}    & + & + & + & + & \textbf{primary}\\ 
    \hline
    \textbf{Pluperfect} & + & - & - & - & \textbf{secondary}\\ 
    \hline
\end{tabular}
\caption{Tense-Mood Paragigms}
\label{tab:tense-mood}
\end{table*}

At this stage, only paradigms in the active voice and indicative mood will be considered,
for the sake of keeping the scope reasonable and for the fact that \textit{all}
tense systems are represented within the paradigms of the indicative mood.

\subsection{Tense and Aspect}

While Greek is usually described as having six ``tenses", these ``tenses" 
are more accurately described as morphological systems designated to a 
tense-aspect pair. Tenses are divided into two ``sequences" (tenses). There
are also three apects--\textit{perfective, imperfective}, and
\textit{stative} \citep{willi2018}.
The six systems (referred to as ``tenses") are yielded by the combination of 
two sequences (\textit{non-past} and \textit{past}) with three aspects. The 
tense aspectual pairs are shown in Table \ref{tab:tense-aspect}.

\begin{table*}
\centering
\begin{tabular}{|c|c|c|c|}
    \hline
    & \textbf{Perfective} & \textbf{Imperfective} & \textbf{Stative}\\
    \hline
    \textbf{non-past} & Future & Present & Perfect\\
    \hline
    \textbf{past} & Aorist & Imperfect & Pluperfect\\
    \hline
\end{tabular}
\caption{Table of the tense-aspect system}
\label{tab:tense-aspect}
\end{table*}

This table will be relevant during formalization, because it represents the
mapping of tense-systems to tense-aspect pairs. As will be seen in the
analysis of an example paradigm, some morphemes are conditioned on the tense
(non-past or past) or the aspect, as opposed to the tense system itself (present,
future, perfect, etc.). Again, what are referred to as ``tenses'' in the Greek
grammatical tradition, are actual tense-apsect pairs, and thus it is preferable
for the formalization to utilize the traditional terminology in the input, map
the ``tense'' to a tense-aspect pair, and then use those features for the
selection of morphemes.

\subsection{Romanization}

This paper will require a consistent system of Romanization to be accessible
to those without knowledge of the Greek writing system. The table of romanization
is given in Tabel \ref{tab:romanization}.

\begin{table}
\centering
\begin{tabular}{|c|c|}
    \hline
    \textsc{Greek} & \textsc{Latin}\\\hline
    α & a\\\hline
    β & b\\\hline
    γ & g\\\hline
    δ & d\\\hline
    ε & e\\\hline
    ζ & z\\\hline
    η & ē\\\hline
    θ & \macronbelow{t}\\\hline
    ι & i\\\hline
    κ & k\\\hline
    λ & l\\\hline
    μ & m\\\hline
    ν & n\\\hline
    ξ & x\\\hline
    ο & o\\\hline
    π & p\\\hline
    ρ & r\\\hline
    σ & s\\\hline
    τ & t\\\hline
    υ & u\\\hline
    φ & \={p}\\\hline
    χ & \macronbelow{k}\\\hline
    ψ & \c{p}\\\hline
    ω & \={o}\\\hline
\end{tabular}
\caption{Romanization scheme}
\label{tab:romanization}
\end{table}

This particular system of Romanization was adapted from that used by Wiktionary
in all of their entries for
Greek.
\footnote{\url{https://en.wiktionary.org/wiki/Wiktionary:Ancient_Greek_transliteration}}
Wiktionary describes their system as \textit{scientific transliteration}--it
is desireable because it is designed for automatic computation, but is also
readable for scholars when examining Greek entries.

Their system uses several digraphs (\textit{ph, th, kh, ps, ks, rh}) to
represent what were originally single graphemes. This is not a problem when
used as a one-way transliteration system for scholarly reference. But for
computation, it is a lot less confusing if there is a one-to-one correspondence
between each Greek letter and its romanization. For this project, several
modifications have been made to Wiktionary's system in order to remove all
digraphs: \textbf{ph, th, kh, ks, rh} are changed to \textbf{\={p},
\macronbelow{t}, \macronbelow{k}, x, r}.  These representations are
inspired from the scholarly transliteration of Arabic from Hans Wehr's Arabic
Dictionary \citep{wehr61}. This system was chosen in order to keep the three
way distinction between voiced, aspirated, and unaspirated stops.

One final digraph not mentioned is the Greek letter Psi \textbf{ψ}, usually
romanized as \textbf{ps}. In order to mark that it is one grapheme instead
of two, it is written as \c{p} (\textit{p} with a cedilla diacritic
beow the letter, representing a little \textit{s}).

In terms of vowels there are the typical five vowels--\textit{a, e, i, o, u}
along with their accompanying long counterparts
(\textit{\={a},\={e},\={i},\={o},\={u}}).

This scheme of romanization is provisional, and only used for the purposes of
this paper. The use of diacritics is meant to reduce the amount of graphemes,
so that each grapheme in Greek is represented by exactly one grapheme
in romanization.

Greek has a complex system of accentuation. In the scope of this paper, this
will not be addressed--this is fine for our purposes, and has been an approach
taken in prior projects of a similar nature \citep{crane91}. However, it will
be a necessary step to address it in future work. For the sake of completeness,
accents are still included in romanization.

\subsection{Description of graphemes}

Graphemes will briefly be presented based on the properties of their respective
phonemes. There are 9 \textit{stops} (also called \textit{plosives}), two
nasals, a liquid, and a silibant. The 9 stops show a three-way distinction based 
on voicing and aspiration. The distribution of consonants is shown in Table
\ref{tab:stops}.

\begin{table*}
\centering
\begin{tabular}{|c|c c c c c c c|}
    \hline
                    & \textbf{unaspirated} & \textbf{aspirated} & \textbf{voiced} & \textbf{nasalized} & \textbf{liquid} & \textbf{trill} & \textbf{silibant}\\\hline
    \textbf{labial} & p                    & \={p}              & b               & m                  &                 &                & \\
    \textbf{dental} & t                    & \macronbelow{t}    & d               & n                  & l               & r              & s\\
    \textbf{velar}  & k                    & \macronbelow{k}    & g               &                    &                 &                & \\\hline
\end{tabular}
\caption{Distribution of cononants (graphemes)}
\label{tab:stops}
\end{table*}

\subsection{Morphemes and Lexemes}

The active-indicative conjugation of \textit{l\'{u}\={o}}, the archetypal
regular verb, is given in table \ref{tab:regular-conjugation}  (the dual number
has been omitted for brevity).

\begin{table}
\centering
\begin{tabular}{|l|l|}
    \hline
    \textbf{Primary} & \textbf{Secondary}\\
    \hline
    \textsc{Present} & \textsc{Imperfect}\\
    \hline
    l\'{u}\={o} & \H{e}luon\\
    l\'{u}eis   & \H{e}lues\\
    l\'{u}ei    & \H{e}lue\\
    l\'{u}omen  & e\symbol{"0313}l\'{u}omen\\
    l\'{u}ete   & e\symbol{"0313}l\'{u}ete\\
    l\'{u}ousi  & e\symbol{"0313}l\'{u}ousi\\
    \hline
    \textsc{Future} & \textsc{Aorist}\\
    \hline
    l\'{u}s\={o} & \H{e}lusa\\
    l\'{u}seis   & \H{e}lusas\\
    l\'{u}sei    & \H{e}luse\\
    l\'{u}somen  & e\symbol{"0313}l\'{u}samen\\
    l\'{u}sete   & e\symbol{"0313}l\'{u}sate\\
    l\'{u}sousi  & e\symbol{"0313}l\'{u}san\\
    \hline
    \textsc{Perfect} & \textsc{Pluperfect}\\
    \hline
    l\'{e}luka      & e\symbol{"0313}lel\'{u}kein\\
    l\'{e}lukas     & e\symbol{"0313}lel\'{u}keis\\
    l\'{e}luke      & e\symbol{"0313}lel\'{u}kei\\
    lel\'{u}kamen   & e\symbol{"0313}lel\'{u}kemen\\
    lel\'{u}kate    & e\symbol{"0313}lel\'{u}kete\\
    lel\'{u}kasi    & e\symbol{"0313}lel\'{u}kesan\\
    \hline
\end{tabular}
\caption{Regular verb conjugation}
\label{tab:regular-conjugation}
\end{table}

\subsubsection{Overview}

The Greek verb consists of \textit{at most} 5 elements (in reality 6, but here
we ignore derivational morphology and focus only on inflectional morphology). 
Notice how in table \ref{tab:regular-conjugation} the presence of the $e$-augment in the \textit{past tenses}, and
its absence in the \textit{non-past} tenses. In addition, take note of the
tense-suffixes present for each aspect (-$\emptyset$- in the \emph{perfective},
-\textit{s}- in the \emph{imperfective}, and -$k$- in the \emph{stative}).

\begin{itemize}
    \item 2 prefixes
        \begin{enumerate}
            \item \textit{e}-augment (also called \textit{epsilon augment})\\
                present-stem \textit{lu-} > imperfect-stem \textit{e\symbol{"0313}lu-}
            \item Reduplication\footnote{Reduplicated consonants undergo
                \textit{deaspiration} if the initial consonant of the verb stem
                is aspireated. e.g. \textit{\macronbelow{t}\'{u}}- $>^{perf.}$
                \textit{te\macronbelow{t}uk}-} of the stem initial consonant with
                epenthetic \textit{e}\\ verb-stem \textit{lu-} > reduplicated
                \textit{lelu-}
        \end{enumerate}
    \item The \textit{verb-stem} as specified in a lexicon. In this case, the
        archetypal verb \textit{lu}- ("to loosen"/"to unbind")
    \item 3 suffixes
        \begin{enumerate}
            \item A \textit{tense suffix}, determined by the aspect
                (perfective, imperfective, or stative)\\
                tense-suffix \textit{-k-} for stative \textit{l\'{e}lu-k-a},
                tense-suffix \textit{-s-} for 1st singular aorist
                \textit{\koronis{e}l\'{u}-s-ate}
            \item A \textit{thematic vowel}, determined by the tense, person, and number.\\
                Thematic vowel -\textit{o}- in 1 singular imperfect \textit{\koronis{e}lu-o-n}
            \item A \textit{personal ending}, determined by the tense, person, and nummber.\\
                Personal ending \textit{-n} in \textit{\koronis{e}luo-n}
        \end{enumerate}
\end{itemize}

For each of the four morphemes, we will define a table/matrix defining which
features condition their usage, and use the defined system to describe an
example paradigm. It is important to note the distinction between the
\textit{verb-stem} and the \textit{tense-stem}. The verb-stem is the smallest
unit with no added features drawn from the lexicon, while the tense-stem consists
of the verb-stem combined with the prefixes (or lack thereof) and \textit{tense-suffix}
(or lack thereof) conditioned by the tense-system (tense-aspect pair). There is a minimum
of three morphemes (verb stem, thematic vowel, and personal ending).

\subsubsection{Prefixes}

The \textit{e}-augment is prefixed to the beginning of the verb stem in the
past-tense (secondary sequence, or the imperfect, aorist, and pluperfect
tenses). Reduplication of the initial consonant occurs in the stative aspect
(perfect and pluperfect tenses). The following table shows their distribution.

\begin{table*}
\centering
\begin{tabular}{|l|l|l|l|}
    \hline
                        & \textbf{Perfective}   & \textbf{Imperfective} & \textbf{Stative}\\
    \hline
    \textbf{non-past}   & $\emptyset$           & $\emptyset$           & reduplication\\
    \hline
    \textbf{past}       & \textit{e}-augment    & \textit{e}-augment    & \textit{e}-augment+reduplication\\
    \hline
\end{tabular}
\caption{Distribution of prefixes}
\label{tab:prefixes}
\end{table*}

This is where considerable economy is obtained in the conditioning of morphemes
when tense and aspect are separated from the traditional \textit{tenses}. It is
clear that each morpheme is conditioned by a single feature (with overlap in
the past stative). But if there was no distinction between tense and aspect,
all of the tense-systems would require morphemes assigned to them individually.

\subsubsection{Suffixes}

The Greek verb has at most three suffixes.

\begin{enumerate}
    \item The \textit{tense-suffix} is a suffix added to the verb-stem conditioned
        on the aspect. The three tense-suffixes are:
        \begin{itemize}
            \item -\textit{s}- for the \textbf{perfective}
            \item -$\emptyset$- for the \textbf{imperfective}
            \item -\textit{k}- for the \textbf{stative}
        \end{itemize}
    \item The \textit{thematic vowel} is infixed between the tense-stem and the
        personal ending. There are three variations of the thematic vowel 
        that are conditioned on the tense-system; \textit{e/o}, \textit{a}, and
        \textit{e}. The distribution of these theme vowels is conditioned by the
        tense-system and the person-number, and is shown below in Table \ref{tab:theme}.
    \item The \textit{personal ending} is conditioned by the person and number, and
        is the final prefix. There are 5 different sets of endings, but we will only
        focus on the \textit{thematic} endings, both \textit{primary} and
        \textit{secondary}. All the endings will be given for the sake of completeness,
        but we are not addressing the middle-passive system in this paper. These are given
        in table \ref{tab:personal-endings}
\end{enumerate}

All of these suffixes can be represented as matrices, the dimensions of which
correspond to each feature conditioning each respective suffix. This will be
defined explicitely in the next section.

\begin{table}
\centering
\begin{tabular}{|c|c|c|c|c|c|c|}
    \hline
    & \textbf{1s} & \textbf{2s} & \textbf{3s} & \textbf{1p} & \textbf{2p} & \textbf{3p}\\
    \hline
    Present     & \multirow{3}{*}{o} & \multicolumn{2}{c|}{\multirow{3}{*}{e}} & \multirow{3}{*}{o} & \multirow{3}{*}{e} & \multirow{3}{*}{o}\\
    Future      &                    & \multicolumn{2}{c|}{}   &                    &                    &                   \\
    Imperfect   &                    & \multicolumn{2}{c|}{}   &                    &                    &                   \\
    \hline
    Aorist      & \multicolumn{6}{c|}{\multirow{2}{*}{a}}\\
    Perfect     & \multicolumn{6}{c|}{}\\
    \hline
    Pluperfect  & \multicolumn{3}{c|}{ei} & \multicolumn{3}{c|}{e}\\ 
    \hline
\end{tabular}
\caption{Theme vowel matrix}
\label{tab:theme}
\end{table}

\begin{table*}
\centering
\begin{tabular}{|c|c|c|c|c|c|}
    \hline
    & \multicolumn{3}{c|}{\textbf{Active}} & \multicolumn{2}{c|}{\textbf{Middle-Passive}}\\ 
    \hline
    & \multicolumn{2}{c|}{\textbf{Primary}} & \multirow{2}{*}{\textbf{Secondary}} & \multirow{2}{*}{\textbf{Primary}} & \multirow{2}{*}{\textbf{Secondary}}\\
    \cline{2-3}
    & \textbf{Thematic} & \textbf{Athematic} & & &\\
    \hline
    \textbf{1sg} & --    & -mi       & -n, --    & -mai  & -mēn\\
    \hline
    \textbf{2sg} & -si   & -s        & -s        & -sai  & -so\\
    \hline
    \textbf{3sg} & -ti    & -si(n)    & --        & -tai  & -to\\
    \hline
    \textbf{1pl} & -men  & -men      & -men      & -meṯa & -meṯa\\
    \hline
    \textbf{2pl} & -te   & -te       & -te       & -sṯe  & -sṯe\\
    \hline
    \textbf{3pl} & -nsi  & -nsi      & -n, -san  & -ntai & -nto\\
    \hline  
\end{tabular}
\caption{Personal endings}
\label{tab:personal-endings}
\end{table*}

\section{Mathematical analysis and formalization}

Now that the conjugation of the regular verb (with our subset of restrictions)
has been described in linguistic terms, it is now necessary to give a
mathematical and computational formulation of all these rules. The following
mathematical analysis is primarily inspired by the approach in
\citet{lambek75}. Each set of morphemes can be viewed as an tuple, conditioned
by cardinality (or indices).

We will first need to define a tuple containg tenses.

\[i = \{nonpast,past\}\]

In tuple $j$, we list all three aspects. As above, these are accessed via
index, or cardinality. Therefore, if $j = 1$, then $j = perfective$.

\[j = \{perfective,imperfective,stative\}\]

Traditionally, tense-aspect combinations are not referred to by a combination
of the names of each respective tense and aspect. Therefore, we define a tuple
$k$ containing the names of each tense-aspect pair. We will provide a mapping between
$i,j$ and $k$ after defining the personal endings.

\[k = \begin{array}{c}
          \{future,\ present,\ perfect,\\
          aorist,\ imperfect,\ pluperfect\}\\
      \end{array}\]

There are 6 possible personal endings. Technically there are 8, 
but the dual number (which is not even present in many dialects,
and never present in the 1st person) has been excluded in order
to reduce the scope of this paper.
      
\[l = \{1s,2s,3s,1p,2p,3p\}\]

Here, a mapping is given for mapping each \emph{tense system} to the respective
tense-aspect pair.

\[
    k = \begin{cases}
            1,1\text{ if }k = 1\\
            1,2\text{ if }k = 2\\
            1,3\text{ if }k = 3\\
            2,1\text{ if }k = 4\\
            2,2\text{ if }k = 5\\
            2,3\text{ if }k = 6
        \end{cases}
\]

The last variable we need to define is the \textbf{stem}. Let $S$ be a tuple,
representing an array of graphemes in the stem. Thus, the graphemes in the stem
$S$ can be accessed with an index. For example, if $S = lu\text{-}$, then $S =
\{l, u\}$m and $S_1
= l$ 

This brings us to the main conjugation function $C$ which takes only a stem as
input.

\begin{equation}
    C_{k,l}(S) = \textsc{Prefix}^{S}_{i,j}+S+\textsc{Suffix}_{k,l}
\end{equation}

The function for generating the prefix takes the stem $S$ and the 
tense aspect pair $(i,j)$ as input. We let $\rho$ represent the
reduplication prefix, and $\alpha$ as the $e$-augmentation prefix.

\begin{equation}
    \textsc{Prefix}^{S}_{i,j} = \alpha_i + \rho_{S,i}
\end{equation}

The conditioning of reduplication $\rho$ is as follows.

\begin{equation}
    \rho_{S,j} = \begin{cases}
                     S_1 \text{e}\text{ if }j=3\\
                     \emptyset\textit{ otherwise}
                 \end{cases}
\end{equation}

The conditioning of the $e$-augment is as follows.

\begin{equation}
    \alpha_i = \begin{cases}
                   \text{\koronis{e} if }i=2\\
                   \emptyset\text{ otherwise}
               \end{cases}
\end{equation}

Finally, the suffixes are obtained from selection of the theme vowel and
personal ending from their respective matrices. These matrices are represented
by tables \ref{tab:theme} and \ref{tab:personal-endings}, respectively.

\begin{equation}
    \textsc{Suffix}_{k,l} = \textsc{Theme}_{k,l} + \textsc{Ending}_{i,l}
\end{equation}

\section{Uses of Current Formalization}

Here is an example derivation, showing the synthesis of the 3rd person
singular, present active indicative form of the archetyple verb
\textit{l\'{u}\={o}} ``to unbind, to loosen".

\[C_{k,l}(S) = \textsc{Prefix}^{S}_{i,j}+S+\textsc{Suffix}_{k,l}\]

Here we substitute the parameters (stem and morphological features)
into their respective variable.

\[\textsc{Prefix}^{\text{lu}}_{1,1}+\text{lu}+\textsc{Suffix}_{2,3}\]

\[\alpha_1\rho_{\text{lu},1}\text{lu}\textsc{Suffix}_{2,3}\]

\[\emptyset\emptyset\text{lu}\textsc{Suffix}_{2,3}\]

When there are \emph{zeros}, they can be eliminated.

\[\text{lu}\textsc{Suffix}_{2,3}\]

\[\text{lu}\textsc{Theme}_{2,3}\textsc{Ending}_{1,3}\]
\[\text{lue}\textsc{Ending}_{1,3}\]
\[\text{lueti}\]

We can finally say that $C_{2,3}(\text{lu}) = \textit{lueti}$. However, there
is an issue--in table \ref{tab:regular-conjugation}, the 3rd person singular
present active indicative form of \textit{l\'{u}\={o}} is \textit{l\'{u}ei},
not *\textit{l\'{u}eti}. 

This discrepency is due to a historical development of Greek phonology, in
which intervocalic -$t$- became -$s$- regularly. This was followed by another
change, in which intervocalic -$s$- was eliminated regularly. With these two
rules in mind, we can illustrate the derivation of the actual form
\textit{l\'{u}ei} from the form that was synthesized by the given functions
(*\textit{l\'{u}eti}).

\centerline{*\textit{l\'{u}eti} > *\textit{l\'{u}esi} > *\textit{l\'{u}ei}}

Since these functions work on the level of \textit{graphemes} and not \textit{phonemes},
derivations (or alternatively called \emph{rewrites}) are labelled as
\textbf{morphographemic rules}. In philological terminology, the form generated
before the application of morphographemic rules is calle the \textit{underlying form},
while the form synthesized after the application of the morphographemic rules is
called the \textit{surface form}. In table \ref{tab:underlying}, the present tense
conjugation of \textit{l\'{u}\={o}} is given, comparing the underlying forms with
the surface forms. Wherever there is a discrepency, the underlying form is marked with
an asterisk.

\begin{table}
\centering
\begin{tabular}{|l l|}
    \hline
    \textbf{Underlying} & \textbf{Surface}\\
    \hline
    *luo$\emptyset$ & lu\={o}\\
    *luesi   & lueis\\
    *lueti   & luei\\
    luomen   & luomen\\
    luete    & luete\\
    *luonsi  & luousi\\
    \hline
\end{tabular}
\caption{Underlying forms compared to surface forms in the present-tense
    conjugation}
\label{tab:underlying}
\end{table}

Transforming the underlying forms into the surface forms, in the past, has
meant hand-writing a very large number of rules and testsing for their
accuracy. 

\section{Conclusion}

Using a system such as the one described in this paper, it is possible to
compose a corpus of \textit{underlying forms} aligned with \textit{surface forms}.
Using statistical methods, the underlying forms generated by conjugation function,
aligned with verb forms from preexisting corpora, provides the potential for
eliminating the laborious step of writing morphographemic rules, with statistical
tools (such as a neural network sequence to sequence model) inferring the rules fom
the given data. It seems that this is a potentially powerful line of inquiry that
should be explored further.

\bibliography{sources}

\end{document}
